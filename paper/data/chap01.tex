% !TeX root = ../thuthesis-example.tex

\chapter{引言}

\section{研究背景}

根据世界卫生组织(WHO)2018年发布的《2018年全球道路安全现状报告》,全球每年交通事故死亡人数达到了135万。交通事故已经成为5至29岁人群的主要死亡原因。对于发展中国家,
道路交通安全形势更加严峻。\cite{WHO} 根据国家统计局发布的数据,2022年我国共发生交通事故2.4万余起,交通事故死亡人数总计6.1万余人,交通事故造成直接财产损失13余
亿元。\cite{stats_gov}

随着科学技术的发展,自动驾驶技术日渐成熟,各大国内外公司和研究机构都表现出了对自动驾驶技术的浓厚兴趣。自动驾驶技术有望减轻驾驶员的驾驶负担、为残障人士和老人提供自动
驾驶服务、提高道路交通安全、改善道路拥挤堵塞情况,并很可能在不远的未来成为影响交通状况、交通安全的重要元素。但自动驾驶汽车的加入,也会引入新的问题,一方面,
随着自动驾驶技术的逐渐普及,可以预见将长期存在自动驾驶车辆和人工驾驶车辆共存的局面,这会使得交通路况更加复杂,引入更多的不确定性因素和安全隐患;
另一方面,目前自动驾驶技术和测试手段仍然不够成熟,自动驾驶相关法律仍在完善之中,法律责任的确定比较模糊,这些问题使得人们仍然对自动驾驶心存顾虑。

近年来,美国电动汽车及能源公司特斯拉因为自动驾驶技术已发生多起安全事故。2018年5月8日,在美国佛罗里达州劳德代尔堡,一辆2014年产的特斯拉汽车撞上混泥土墙并起火,造成
2名高中生死亡,另有一名高中人受伤;2021年5月7日,在中国广东省韶关市,一辆特斯拉汽车追尾一辆小型货车,造成前者驾驶人当场死亡。这些事故使得人们越来越关注自动驾驶的安全问题。

车队的控制有一个特殊的困难,称为“队列不稳定性”,即系统中的扰动沿着车队不断放大。自动驾驶技术的引入,使得人类可以更加精准地控制车辆,通过控制手段,可以使车队达到“稳定”。
围绕车队稳定性,已经有非常丰富的研究。

直观感受上,车队中车辆在速度和位置上的波动是造成车辆碰撞的主要原因,前者可以用车队的队列稳定性来描述,后者可以用碰撞风险来描述,如果能将二者建立联系,就能够用车队的队列稳定性
预测碰撞风险,同时可以通过控制车队的队列稳定程度达到减小碰撞概率的目的,提高道路交通安全程度。

本研究旨在建立自动驾驶车辆和人工驾驶车辆混合车队情景下队列稳定性与碰撞风险的关系,探究碰撞风险在车队中的演化机理,为车队的安全性分析提供理论基础。




\section{国内外研究现状综述}

\subsection{跟驰行为建模研究现状}

驾驶员的驾驶行为和所产生的车辆运行特征是研究交通流的基础。而非自由驾驶情况下,车辆直接的相互作用和导致的交通流变化则是研究的重点。

车辆运动行为主要可以分为车辆跟驰行为和换道行为两大类,跟驰模型的研究对象是前者。用数学模式对跟驰行为加以分析阐明,使得研究人员可以定量地描述跟驰行为,对现代交通的模拟有着重要的意义。
跟驰模型研究主要是运用动力学、统计学等方法,利用驾驶行为问卷调查、模拟驾驶或自然驾驶实验的方式,对前车速度、加速度等行车特征变化引起后车的反应进行研究。

研究者提出了多种跟驰模型,下面分别进行简要介绍。

\begin{description}
  \item [(1)刺激-反应跟驰模型]
    \begin{equation*}
      \Gamma \Delta \Theta \Lambda \Xi \Pi \Sigma \Upsilon \Phi \Psi \Omega.
    \end{equation*}
    注意有限增量符号 $\increment$ 固定使用正体,模板提供了 \cs{increment} 命令。
  \item 小于等于号和大于等于号使用倾斜的字形 $\le$、$\ge$。
  \item 积分号使用正体,比如 $\int$、$\oint$。
  \item 行间公式积分号的上下限位于积分号的上下两端,比如
    \begin{equation*}
      \int_a^b f(x) \dif x.
    \end{equation*}
    行内公式为了版面的美观,统一居右侧,如 $\int_a^b f(x) \dif x$ 。
  \item
    偏微分符号 $\partial$ 使用正体。
  \item
    省略号 \cs{dots} 按照中文的习惯固定居中,比如
    \begin{equation*}
      1, 2, \dots, n \quad 1 + 2 + \dots + n.
    \end{equation*}
  \item
    实部 $\Re$ 和虚部 $\Im$ 的字体使用罗马体。
\end{description}

\subsection{车队稳定性分析研究现状}

\subsection{车队碰撞风险评估指标研究现状}

\subsection{研究现状总结}

\section{课题研究目标与难点}

\subsection{研究目标}

\subsection{研究难点}




\section{论文结构及章节安排}

一篇学位论文的引言大致包含如下几个部分:
1、问题的提出;
2、选题背 景及意义;
3、文献综述;
4、研究方法;
5、论文结构安排。
\begin{itemize}
  \item 问题的提出:要清晰地阐述所要研究的问题“是什么”。
    \footnote{选题时切记要有“问题意识”,不要选不是问题的问题来研究。}
  \item 选题背景及意义:论述清楚为什么选择这个题目来研究,即阐述该研究对学科发展的贡献、对国计民生的理论与现实意义等。
  \item 文献综述:对本研究主题范围内的文献进行详尽的综合述评,“述”的同时一定要有“评”,指出现有研究状态,仍存在哪些尚待解决的问题,讲出自己的研究有哪些探索性内容。
  \item 研究方法:讲清论文所使用的学术研究方法。
  \item 论文结构安排:介绍本论文的写作结构安排。
\end{itemize}



