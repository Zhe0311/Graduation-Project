% !TeX root = ../thuthesis-example.tex

% 中英文摘要和关键字

\begin{abstract}
随着自动驾驶技术的日渐成熟,自动驾驶车辆和人工驾驶车辆混行很有可能成为未来道路上的常见场景。
在本课题中,首先为自动驾驶和人工驾驶车辆选择了合适的跟驰模型,并对混合车队的队列稳定性进行了分析,
接着通过数值仿真的方法,探究了混合车队的队列稳定性与碰撞风险之间的关系,同时探讨了碰撞风险在混合车队中的演化机理。

本工作的意义在于,碰撞风险的评价指标的获得往往是滞后于危险发生的,这不利于交通事故的预测与防范。如果能建立混合车队
队列稳定性与碰撞风险的关系,就可以通过队列的稳定性对车队的碰撞风险进行评估,起到预防交通事故的作用。同时,通过对碰撞风险
演化机理的探究,发现混合车队的排列情况也会影响车队的碰撞风险,其中的机理和规律可以指导车队选择碰撞风险更小的排列方式。


  % 关键词用“英文逗号”分隔,输出时会自动处理为正确的分隔符
  \thusetup{
    keywords = {自动驾驶, 混合车队, 队列稳定性, 碰撞风险},
  }
\end{abstract}

\begin{abstract*}
  
With the development of autonomous driving technology, 
the mixed driving of autonomous vehicles and human-driven vehicles is likely to become a common situation in the future.
In this study, the appropriate car-following models are firstly selected for autonomous and human-driven vehicles,
and the string stability of the mixed vehicular platoon is analyzed.
Besides, the relationship between the string stability and the crash risk, 
as well as the evolution mechanism of the crash risk,  of the mixed vehicular platoon is discussed 
through the situation. 

The significance of this work is that the acquisition of the evaluation of crash risk usually lags behind the
occurrence of danger, which is not conducive to the prediction and prevention of traffic accidents.
If we can build the relationship between the string stability and the crash risk of the mixed vehicular platoon,
we can use the former to predict the latter,
which plays a role in preventing traffic accidents. 
At the same time, the arrangement of the mixed vehicular platoon also affects the crash risk,
and it is significant to study the evolution mechanism of the crash risk 
to guide the mixed vehicular platoon to choose the arrangement with less crash risk.

  % Use comma as separator when inputting
  \thusetup{
    keywords* = {automated vehicles, mixed vehicular platoon, string stability, crash risk},
  }
\end{abstract*}