% !TeX root = ../thuthesis-example.tex

\chapter{总结与展望}

\section{课题总结}

本课题基于数值仿真的方法,探究了自动驾驶和人工驾驶混合车队的队列稳定性与碰撞风险之间的关系,以及碰撞风险的演化机理。

本课题主要完成了以下几项工作:

一、仿真平台的搭建。首先选取了合适的人工驾驶车辆和自动驾驶车辆的跟驰模型,基于MATLAB完成了$500s$仿真流程的建立,仿真的结果包含如下
信息:
\begin{itemize}
    \item 各车的速度、加速度、位移:方便后续的统计分析工作
    \item 碰撞信息:包括是否发生碰撞、发生碰撞的时间、此时的车队排列、碰撞车辆的下标,发生碰撞后会立即停止当前实验,进入下一个实验
    \item 队列稳定性:对于发生碰撞的样本,返回$G_{max}$指标(参见\ref{sec:4.1}节),对于未发生碰撞的样本,返回$t_{stable}$指标
    (参见\ref{sec:4.3}节)
    \item 潜在危险时间比例:对于未发生碰撞的样本,返回潜在危险时间比例
\end{itemize}

该仿真平台的一大特点是考虑了很多实际因素,比如加速度大小限制、人工驾驶员反应延迟,使得仿真环境更加接近真实环境。

二、混合车队的队列稳定性分析。首先基于人工驾驶车辆和自动驾驶车辆的跟驰模型推导了二者对速度扰动的传递函数,接着基于队列稳定性(String Stability)的
概念得到了对混合车队稳定性的评估方法,最后通过仿真的方式对该方法进行了验证。

三、混合车队队列稳定性与碰撞风险关系的探究。对于碰撞样本和非碰撞样本分别选取了队列稳定性指标与碰撞风险指标,通过仿真和统计的方式建立了
稳定性与碰撞风险的关系,并对仿真结果的统计数据呈现的规律进行了物理意义上的分析。

四、混合车队碰撞风险演化机理的探究。通过改变混合车队的空间分布,观察碰撞风险的变化情况,以得到碰撞风险沿着车队的传播情况,即碰撞风险的演化机理。碰撞风险
的演化机理对如何安全地排列混合车队有指导意义。

\section{课题创新点}

本课题的创新点主要有以下两点:

一、建立了混合车队队列稳定性与碰撞风险的关系。在已有的工作中,关于车队队列稳定性的工作,多是将其与交通拥堵建立联系。而在本课题中则关注的是车队队列稳定性与
碰撞风险之间的关系,此工作的意义在于,队列的稳定性是可以通过计算得到的,而碰撞风险却不能事先预估,建立了二者的关联后,可以用稳定性预测碰撞风险,以起到实现预警的作用。

二、探究了混合车队碰撞风险的演化机理。由于混合车队中含有多于一种的跟驰模型,不同的跟驰方式对扰动的传递效果也是不同的,对于一个车队,不同的排列方式也会有不同的扰动的传递情况,
同样,也会有不同的碰撞风险演化情况。通过对混合车队碰撞风险演化机理的研究,能够了解不同种类的跟驰车辆以及不同的车队排列方式对碰撞风险的影响,对如何提高混合车队的安全性有指导价值。

\section{研究展望}

本课题的重点是建立混合车队队列稳定性与安全性的关系。从实际应用角度来看,这项工作非常有意义,但由于时间仓促以及本人能力有限,
本工作仍有很多不足,也有很多可以继续探究的方向,以下是对后续工作的展望:

一、细化场景划分。本工作中进行了$10m/s$到$30m/s$的速度区间的仿真工作,换算为生活中常用的汽车速度单位为$36km/h$到$108km/h$,这覆盖了市区、郊区、国道等
场景的速度区间,但不同的场景可能有不同的驾驶模式,如果能分场景探讨稳定性和安全性的关系,以及碰撞风险的演化机理,可能更加贴近真实情况。

二、建立更加完善的基于队列稳定性的安全评价体系。本工作是基于已经选定的人工驾驶和自动驾驶跟驰模型展开的,且二者的跟驰模型参数是固定的,在不同的场景下(如市区、郊区、高速)模型参数的取值会有变化,
这也是本人认为本工作的一个不足之处。同理,跟驰模型也不是适用于所有场景的。虽然如此,但是本工作并不是没有意义,第一,本工作中用到的分析方法是可以推广到所有场景的,比如车队队列稳定性
的分析方法,以及碰撞风险指标的选取,在很多情况下都是成立的。换句话说,即使使用了其他的跟驰模型,即使改变了跟驰模型的参数,也可以用同样的方法进行分析,基于这样的想法,可以将本工作的分析方法抽离出来,
建立一套基于队列稳定性的安全评价体系,以分析更加复杂的、任意跟驰模型的、任意跟驰模型参数的车队。第二,本工作中得到的一些关于混合车队稳定性与碰撞风险的关系以及混合车队碰撞风险的演化机理,是有普适性的。

三、由于自动驾驶车辆尚未大范围投入使用,能难获得真实交通环境下混合车队的数据,因此没有用真实交通数据对本工作的结论进行验证。希望未来能有机会对仿真结论进行验证,或者用真实数据进行车队稳定性与碰撞风险
的关系以及碰撞风险演化机理的探究。